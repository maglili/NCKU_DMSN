\documentclass{article}

% if you need to pass options to natbib, use, e.g.:
     \PassOptionsToPackage{numbers, compress}{natbib}
% before loading neurips_2021

% ready for submission
%\usepackage{neurips_2021}

% to compile a preprint version, e.g., for submission to arXiv, add add the
% [preprint] option:
     \usepackage[preprint]{neurips_2021}

% to compile a camera-ready version, add the [final] option, e.g.:
%     \usepackage[final]{neurips_2021}

% to avoid loading the natbib package, add option nonatbib:
%    \usepackage[nonatbib]{neurips_2021}

\bibliographystyle{unsrtnat} % <----------- must add for natbib 

\usepackage[utf8]{inputenc} % allow utf-8 input
\usepackage[T1]{fontenc}    % use 8-bit T1 fonts
\usepackage{hyperref}       % hyperlinks
\usepackage{url}            % simple URL typesetting
\usepackage{booktabs}       % professional-quality tables
\usepackage{amsfonts}       % blackboard math symbols
\usepackage{nicefrac}       % compact symbols for 1/2, etc.
\usepackage{microtype}      % microtypography
\usepackage{xcolor}         % colors

\title{DMSN final project: Improve LESSR model structure}

% The \author macro works with any number of authors. There are two commands
% used to separate the names and addresses of multiple authors: \And and \AND.
%
% Using \And between authors leaves it to LaTeX to determine where to break the
% lines. Using \AND forces a line break at that point. So, if LaTeX puts 3 of 4
% authors names on the first line, and the last on the second line, try using
% \AND instead of \And before the third author name.

\author{%
  TENG, LI-CHANG\\
  Department of Electrical Engineering\\
  National Cheng Kung University\\
  \texttt{n26091194@gs.ncku.edu.tw} \\
  % examples of more authors
  \And
  TENG, LI-CHANG\\
  Department of Electrical Engineering\\
  National Cheng Kung University\\
  \texttt{n26091194@gs.ncku.edu.tw} \\
  \AND
  TENG, LI-CHANG\\
  Department of Electrical Engineering\\
  National Cheng Kung University\\
  \texttt{n26091194@gs.ncku.edu.tw} \\
  \And
  TENG, LI-CHANG\\
  Department of Electrical Engineering\\
  National Cheng Kung University\\
  \texttt{n26091194@gs.ncku.edu.tw} \\
%   \And
%   Coauthor \\
%   Affiliation \\
%   Address \\
%   \texttt{email} \\
}

\begin{document}

\maketitle

%%%%%%%%%%%%%%%%%%%%%%%%%%%%%%%%%%%%%%%%%%%%%%%%%%%%%%%%%%%%

\begin{abstract}
    None
    \vspace{2cm}
\end{abstract}

%%%%%%%%%%%%%%%%%%%%%%%%%%%%%%%%%%%%%%%%%%%%%%%%%%%%%%%%%%%%

\section{INTORDUCTION}
None
\vspace{3cm}

%%%%%%%%%%%%%%%%%%%%%%%%%%%%%%%%%%%%%%%%%%%%%%%%%%%%%%%%%%%%

\section{RELAED WORK}
None
\vspace{3cm}

%%%%%%%%%%%%%%%%%%%%%%%%%%%%%%%%%%%%%%%%%%%%%%%%%%%%%%%%%%%%

\section{PRELIMINARY}
None
\vspace{3cm}


%%%%%%%%%%%%%%%%%%%%%%%%%%%%%%%%%%%%%%%%%%%%%%%%%%%%%%%%%%%%

\section{EXPERIMENTS}

In this section, we will introduce experiment setting,
dataset, and analyze the experiment result.
We conducted several experiments to check out hypotheses and evaluate our
model with choosen metric.

\subsection{Dataset}

We choose Diginetica dataset\footnote{\url{https://competitions.codalab.org/competitions/111610}}
following LESSR \cite{chen2020lessr} paper,
which is the CIKM cup 2016 dataset provided by DIGINETICA Crop.
There are 6 files in Diginetica dataset, but we only need the transaction one.
As \cite{chen2020lessr}, we used last week sessions as test data.
We got the same training and test set by following preprocessing method
described in \cite{chen2020lessr}.
Statistics of Diginetica dataset is shown in Table~\ref{data-stats}.

\begin{table}
    \caption{statistics of dataset}
    \label{data-stats}
    \centering
    \begin{tabular}{ll}
        \toprule
        \multicolumn{2}{c}{Diginetica} \\
        \midrule
        No. of Clicks   & 981,620      \\
        No. of Sessions & 777,029      \\
        No. of Items    & 42,596       \\
        Average length  & 4.80         \\
        \bottomrule
    \end{tabular}
\end{table}

\subsection{Baseline and metrics}

We choose \cite{chen2020lessr} as out baseline,
then we tried to improve model structure in \cite{chen2020lessr} by some changes.
Comparing the metrics to \cite{chen2020lessr},
we could know the change is postive or negative influence.
Following \cite{chen2020lessr},
the metrics we used are HR@20 (Hit Rate) and MRR@20 (Mean Reciprocal Rank).

\subsection{Multi-Head Attention}

MUTIHEADATTENTION\footnote{\url{https://pytorch.org/docs/stable/generated/torch.nn.MultiheadAttention.html}}
is a official implemented self-attention layer by pytorch.
Here we replace GRU\footnote{\url{https://pytorch.org/docs/stable/generated/torch.nn.GRU.html}}
layer in EOPA block in \cite{chen2020lessr} by MUTIHEADATTENTION layer.
All settings are the same but GRU now replaced by MUTIHEADATTENTION.
We adjusted num of heads parameter in MUTIHEADATTENTION layer
to see the influence of muti-head attention.

The pytorch official did not implemented positional encoding in
MUTIHEADATTENTION layer, so there is no position information within layer.
To handle this problem we need to do position encoding manually.
We fonud a offical tutorial\footnote{\url{https://pytorch.org/tutorials/beginner/transformer_tutorial.html}}
that manually implemented position encoding,
so we followed the encoding method here.

Table~\ref{Multi-head w/o pos encoding} and Table~\ref{Multi-head with pos encoding}
are the experiment result without and with positional encoding respectively.
It turns out no matter multi-head or positional encoding, can not improve the result.
So in next section we decided to using more complex layer.

\begin{table}
    \caption{Multi-head w/o pos encoding}
    \label{Multi-head w/o pos encoding}
    \centering
    \begin{tabular}{lllll}
        \toprule
        AGG.TYPE & HR@20 & MRR@20 & NDCG@20 & Total impv.  \\
        \midrule
        baseline & 52.82 & 18.3   & 25.93   & -            \\
        Head=1   & 52.65 & 18.25  & 25.85   & -0.903594775 \\
        Head=2   & 52.58 & 18.27  & 25.84   & -0.965396084 \\
        Head=4   & 52.6  & 18.28  & 25.85   & -0.834321462 \\
        Head=8   & 52.63 & 18.28  & 25.87   & -0.700394057 \\
        Head=16  & 52.62 & 18.28  & 25.86   & -0.757891648 \\
        Head=32  & 52.64 & 18.29  & 25.87   & -0.626817026 \\
        \bottomrule
    \end{tabular}
\end{table}

\begin{table}
    \caption{Multi-head with pos encoding}
    \label{Multi-head with pos encoding}
    \centering
    \begin{tabular}{lllll}
        \toprule
        AGG.TYPE & HR@20 & MRR@20 & NDCG@20 & Total impv.  \\
        \midrule
        baseline & 52.82 & 18.3   & 25.93   & -            \\
        Head=1   & 52.57 & 18.26  & 25.83   & -1.077538484 \\
        Head=2   & 52.55 & 18.29  & 25.85   & -0.874337766 \\
        Head=4   & 52.59 & 18.3   & 25.88   & -0.628267962 \\
        Head=8   & 52.62 & 18.29  & 25.87   & -0.664681471 \\
        Head=16  & 52.54 & 18.31  & 25.86   & -0.745415003 \\
        Head=32  & 52.57 & 18.32  & 25.89   & -0.518277422 \\
        \bottomrule
    \end{tabular}
\end{table}

\subsection{Transformer Encoder}

In this section, we use transformer encoder
\footnote{\url{https://pytorch.org/docs/stable/generated/torch.nn.TransformerEncoder.html}}
to replace GRU.
TransformerEncoder has a lot of hyperparameter, so we conducted
3 main experiments to tuning the model:
1) dim\_feedforward 2) nhead 3) encoder\_layer.
Also, each main experiments have two sub experiments:
1) w/o pos encoding 2) with pos encoding.

\subsubsection{Dim\_Feedforward Experiment}

In this experiment we fix all hyperprameters but dim\_feedforward.
Table~\ref{dim exp w/o pos encoding} shown the result without
positional encoding.
Table~\ref{dim exp with pos encoding} shown the result with positional encoding.

From Table~\ref{dim exp w/o pos encoding} and Table~\ref{dim exp with pos encoding},
we fonud that the best dim\_feedforward is seeting 512,
whether with pos encoding or not, dimension 512 in both case has a good result,
so we choose dimension 512 for our model in the later experiments.

\begin{table}
    \caption{dim exp w/o pos encoding}
    \label{dim exp w/o pos encoding}
    \centering
    \begin{tabular}{lllll}
        \toprule
        AGG.TYPE   & HR@20 & MRR@20 & NDCG@20 & Total impv.  \\
        \midrule
        baseline   & 52.82 & 18.3   & 25.93   & -            \\
        Dim = 2048 & 52.73 & 18.25  & 25.83   & -0.82926773  \\
        Dim = 1024 & 52.9  & 18.31  & 25.86   & -0.063854988 \\
        Dim = 512  & 52.97 & 18.35  & 26      & 0.827164961  \\
        Dim = 256  & 52.67 & 18.28  & 25.88   & -0.586099799 \\
        Dim = 128  & 52.88 & 18.34  & 25.94   & 0.370737939  \\
        Dim = 64   & 52.84 & 18.39  & 26.01   & 0.83819067   \\
        Dim = 32   & 52.7  & 18.24  & 25.85   & -0.863578471 \\
        Dim = 16   & 52.68 & 18.3   & 25.88   & -0.457877958 \\
        \bottomrule
    \end{tabular}
\end{table}

\begin{table}
    \caption{dim exp with pos encoding}
    \label{dim exp with pos encoding}
    \centering
    \begin{tabular}{lllll}
        \toprule
        AGG.TYPE   & HR@20 & MRR@20 & NDCG@20 & Total impv.  \\
        \midrule
        baseline   & 52.82 & 18.3   & 25.93   & -            \\
        Dim = 2048 & 52.74 & 18.28  & 25.88   & -0.45357424  \\
        Dim = 1024 & 52.86 & 18.3   & 25.88   & -0.117097951 \\
        Dim = 512  & 52.85 & 18.36  & 25.97   & 0.538926994  \\
        Dim = 256  & 52.7  & 18.26  & 25.86   & -0.715723485 \\
        Dim = 128  & 52.89 & 18.37  & 25.95   & 0.592169956  \\
        Dim = 64   & 52.79 & 18.34  & 25.95   & 0.238913304  \\
        Dim = 32   & 52.74 & 18.29  & 25.9    & -0.321798695 \\
        Dim = 16   & 52.6  & 18.36  & 25.92   & -0.127205414 \\
        \bottomrule
    \end{tabular}
\end{table}

\subsubsection{Multi-Head Experiment}

Here we fixed all hyperprameters but nhead to see the influence.
Also, The dim\_feedforward set to 512.
Result without positional encoding shown in
Table~\ref{multi-head exp w/o pos encoding} and result with
positional encoding shown in Table~\ref{multi-head exp with pos encoding}.

Comparing  Table~\ref{multi-head exp w/o pos encoding} and
Table~\ref{multi-head exp with pos encoding},
We found the metrics without positional encoding are usually
better than the other one.
So positional information might not a critical info in this scenario.

Note that best performance appeared when nhead set to 8 with postional encoding.

\begin{table}
    \caption{multi-head exp w/o pos encoding}
    \label{multi-head exp w/o pos encoding}
    \centering
    \begin{tabular}{lllll}
        \toprule
        AGG.TYPE & HR@20 & MRR@20 & NDCG@20 & Total impv. \\
        \midrule
        baseline & 52.82 & 18.3   & 25.93   & -           \\
        nhead=1  & 52.97 & 18.35  & 26      & 0.827164961 \\
        nhead=2  & 52.77 & 18.37  & 25.95   & 0.364983285 \\
        nhead=4  & 52.98 & 18.35  & 26      & 0.846097184 \\
        nhead=8  & 52.87 & 18.37  & 25.96   & 0.592870879 \\
        nhead=16 & 52.78 & 18.37  & 25.97   & 0.461046244 \\
        nhead=32 & 52.92 & 18.41  & 25.97   & 0.944676596 \\
        \bottomrule
    \end{tabular}
\end{table}

\begin{table}
    \caption{multi-head exp with pos encoding}
    \label{multi-head exp with pos encoding}
    \centering
    \begin{tabular}{lllll}
        \toprule
        AGG.TYPE & HR@20 & MRR@20 & NDCG@20 & Total impv. \\
        \midrule
        baseline & 52.82 & 18.3   & 25.93   & -           \\
        nhead=1  & 52.85 & 18.36  & 25.97   & 0.538926994 \\
        nhead=2  & 52.7  & 18.31  & 25.91   & -0.2496726  \\
        nhead=4  & 52.88 & 18.38  & 25.98   & 0.743578647 \\
        nhead=8  & 52.87 & 18.41  & 26.03   & 1.081407692 \\
        nhead=16 & 52.82 & 18.35  & 25.96   & 0.388920149 \\
        nhead=32 & 52.75 & 18.35  & 25.95   & 0.217829222 \\
        \bottomrule
    \end{tabular}
\end{table}

\subsubsection{Num-Layers Experiment}

Here all hyperprameters was fixed but num\_layers will be change.
The dim\_feedforward was set to 512 and nhead was set to 1.
Table~\ref{num-layers exp w/o pos} and Table~\ref{num-layers exp with pos}
shown the result without and with positional encoding respectively.

We found that whether transformer encoder with positional encoding or not,
it has similar trend that performance decreased as layers increased.
And we found as layers increased, models's inference time also increased.

\begin{table}
    \caption{num-layers exp w/o pos}
    \label{num-layers exp w/o pos}
    \centering
    \begin{tabular}{lllll}
        \toprule
        AGG.TYPE & HR@20 & MRR@20 & NDCG@20 & Total impv.  \\
        \midrule
        baseline & 52.82 & 18.3   & 25.93   & -            \\
        layer=1  & 52.97 & 18.35  & 26      & 0.827164961  \\
        layer=2  & 52.72 & 18.41  & 25.93   & 0.41177067   \\
        layer=3  & 52.69 & 18.4   & 25.95   & 0.37745993   \\
        layer=4  & 52.69 & 18.33  & 25.89   & -0.236445941 \\
        layer=6  & 52.65 & 18.13  & 25.72   & -2.060682268 \\
        layer=8  & 52.24 & 17.96  & 25.48   & -4.691433984 \\
        layer=16 & 52.11 & 17.77  & 25.34   & -6.515719401 \\
        \bottomrule
    \end{tabular}
\end{table}

\begin{table}
    \caption{num-layers exp with pos}
    \label{num-layers exp with pos}
    \centering
    \begin{tabular}{lllll}
        \toprule
        AGG.TYPE & HR@20 & MRR@20 & NDCG@20 & Total impv.  \\
        \midrule
        baseline & 52.82 & 18.3   & 25.93   & -            \\
        layer=1  & 52.85 & 18.36  & 25.97   & 0.538926994  \\
        layer=2  & 52.77 & 18.41  & 25.96   & 0.622127888  \\
        layer=3  & 52.72 & 18.34  & 25.88   & -0.163569833 \\
        layer=4  & 52.84 & 18.32  & 25.9    & 0.031457958  \\
        layer=6  & 52.8  & 18.18  & 25.78   & -1.272082675 \\
        layer=8  & 52.3  & 17.95  & 25.46   & -4.709616194 \\
        layer=16 & 52.04 & 17.81  & 25.37   & -6.313969619 \\
        \bottomrule
    \end{tabular}
\end{table}

%%%%%%%%%%%%%%%%%%%%%%%%%%%%%%%%%%%%%%%%%%%%%%%%%%%%%%%%%%%%

\section{CONCLUSION}

In our experimets, no matter with postional encoding or not,
the performance of MUTIHEADATTENTION is worse than GRU,
although using MUTIHEADATTENTION is slightly fast.
In muti-head experiment of TransformerEncoder,
we found there is no distinct trend about num of head,
also we found when TransformerEncoder stacking more layers,
model's performance will decreased, this means the model is over-fitting,
moreover, evaluate time will increased, this cause training time getting longer.

After several experimets, we found that TransformerEncoder can improve
model performance by replacing GRU in EOPA layer,
but the parameter of TransformerEncoder need in proper setting.
In SGAT and Readout layer, we tried to change the attention mechanisms,
but we didn't got a good result.
This might bacause attention mechanisms is dependent on model structure and data,
it won't be useful if we just replace attention mechanisms from different paper.

Finally the best result we got is when layer=1 and nhead=8
with positional encoding, total improvement of three metrics is  1.08\%,
althought there is still a space for improvements in this model.
The result is a acceptable to us because we didn't change the model structure
so much but just change a layer inside EOPA block.
If we could modify SGAT and Readout layer's attention mechanisms
to multi-head attention, we might get a better result.


%%%%%%%%%%%%%%%%%%%%%%%%%%%%%%%%%%%%%%%%%%%%%%%%%%%%%%%%%%%%


\bibliography{refer}


\end{document}